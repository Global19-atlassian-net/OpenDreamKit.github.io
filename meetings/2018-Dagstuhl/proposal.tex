\documentclass[a4paper,11pt]{article}
\usepackage[top=2.5cm,bottom=2.5cm,left=2.5cm,right=2.5cm]{geometry}

\usepackage{url,xcolor,wrapfig,amsmath,amssymb,paralist,xspace,preamble}
\usepackage{tabularx}
\usepackage[show]{ed}

\setcounter{tocdepth}{1}
\usepackage[bookmarks,linkcolor=blue,citecolor=blue,urlcolor=gray,colorlinks,breaklinks,bookmarksopen,bookmarksnumbered]{hyperref}

\setlength{\hfuzz}{3pt}
\hbadness=10001 %make box warning less strict

\title{Application for a Dagstuhl Seminar \\
  \emph{Math-in-the-Middle: An API for Mathematics}}
\author{Jacques Carette and Wolfram Decker and Viviane Pons and Florian Rabe}
\date{\today}


\begin{document}
\maketitle

\begin{abstract}
\ednote{@all revise}

The last decades have witnessed the emergence of a complex ecosystem of open
source software for scientific and mathematical computation, developed by
overlapping international communities of researchers in both computer science
and mathematics, teachers, engineers, and amateurs.

This is part of the greater trend for open and reproducible science.

Current systems range from specialized libraries (e.g. \MPIR, \Linbox) to
thematic systems (e.g. \GAP, \Pari, \Singular, xcas) to general purpose systems
like Mathemagix or \Sage.

Moreover there is a wealth of databases of mathematical objects like the
Online Encyclopedia of Integer Sequences (OEIS), the L-functions and
Modular forms Database (LMFDB), the Small Groups Library, the Online Atlas of
Finite Simple Groups, and many more. 

Interactive computing environments \Jupyter) or knowledge management systems \MathHub.

These very dynamic developments are making it more and more difficult to connect
domain specific, optimised systems into higher-level systems and
avoid the expensive duplicate development of libraries.

Numerous collaborations and
projects to design knowledge representation languages and frameworks aimed at
system interoperability and integration have been initiated, OpenMath, Science,
OpenDreamKit, OSCAR, to name a few.

None of them has reached the critical mass to assemble
significant enough a share of the community to gain the universal appeal that an
interoperability solution needs to gain traction.

In this seminar, we bring together agents from all relevant communities in order
to develop tools, best practices, and open standards for sharing algorithms and
data across computational mathematics systems.
\ednote{I don't want to artificially limit this to ``computer algebra'', this
  should well be wider, like computational mathematics, but not too wide at the
  risk of trying to do too much}
\end{abstract}

\ednote{deadline is April 15th for the Initial Submission, April 22nd for the
  final (Markus check again!)}
\ednote{See \url{http://www.dagstuhl.de/programm/dagstuhl-seminare/antrag/} for details about how to write proposal}

\tableofcontents

\newpage

\ednote{Comment by JC: I don't think it is necessary to re-focus, but rather to
make sure that there are clear "new" topics to discuss, with a clear hope that
the discussion can lead to real progress. So if the workshop is to be about
interoperability of (some) systems that focus on 'computation', it seems that a
good starting point would be a list, articulated by some of the principal
stakeholders of said systems, of what kinds of computations they would like to
borrow from another system AND what currently prevents them from just doing it.

I, of course, entirely agree with the basic idea "that systems should share
knowledge better". But you and I are 'solution providers' in this area, where
the solutions are implemented in systems which otherwise do not "do" enough to
be attractive at large. So we need to work with the system builders to
understand what their actual problems are, and bridge the gap. Or so I think. It
seems we need equal involvement (in the workshop definition) from some of the
principal system builders involved in OpenDreamKit to get a balanced proposal. }

% % % % % % % % % % % % % % % % % %
\section{Basic Information about the Seminar}

\subsection{Title}

The title \emph{Integrating Computation Systems} means to emphasize the vision
of a universal environment for scientific and mathematical computation. We hope
this will provide a joint goal for the currently disparate communities.
\ednote{markus: I find this title a bit bland. I am a fan of catchy titles though.}

% % % % % % % % % % % % % % % % % %
\subsection{Organizers}

\ednote{@all: add yourself}
\begin{itemize}
\item PD Dr. Florian Rabe \\
Jacobs University Bremen - Computer Science\\
P.O. Box 750 561
28725 Bremen Germany
\\
Phone: + 49 421 200 3051\\
%Fax: + 49 421 200 493051\\
Email: \texttt{f.rabe@jacobs-university.de}  \\
Homepage: \url{http://kwarc.info/people/frabe/}

\item PD Dr. Viviane Pons \\
Universit\'e Paris-Sud - Laboratoire de Recherche en Informatique\\
Bat 650 Ada Lovelace
91405 Orsay Cedex France 
\\
Email: \texttt{viviane.pons@lri.fr}  \\
Homepage: \url{https://www.lri.fr/~pons/}
\end{itemize}

% % % % % % % % % % % % % % % % % %
\subsection{Type of event, duration, and size}

We propose a 5-day Dagstuhl Seminar with 45 participants.

\subsection{Topics}

\ednote{@all: we have to select 1-3 from a list of topics; strangely none fits
  very well\\ markus: I commented out the ones that fit least, cutting it down
  to 3 that seem to fit reasonably well}
\begin{compactitem}
 \item databases / information retrieval
% \item data structures / algorithms / complexity
% \item modelling / simulation
 \item semantics / formal methods
 \item sw-engineering
% \item verification / logic
\end{compactitem}

% % % % % % % % % % % % % % % % % %
\subsection{Keywords}

\ednote{@all: revise/extend}
\begin{compactitem}
\item computer algebra
\item computation
\item reusability
\item interoperability
\end{compactitem}

% % % % % % % % % % % % % % % % % %
\subsection{Proposed Seminar Dates}

\ednote{@all: Dagstuhl will pick the date, typically early 2018, but it could be be several months earlier or later; please add your preferences}

Block-out Dates: 
\begin{itemize}
\item 
\end{itemize}

\noindent
Preferred Dates: 
 \begin{itemize}
 \item 
\end{itemize}

% % % % % % % % % % % % % % % % % %
\section{Description of the Seminar}
  
% 2.1 Description of the seminar (3-5 pages, in English):
% - Brief, general introduction to the topic
% - In-depth description of the topic
% - Questions and issues addressed by the seminar; objectives and results
%   expected to be produced by the seminar
% - As applicable: relationship to previous seminars or how the proposed seminar
%   differs from similar seminars. Which new developments and issues are to be
%   addressed? (Tip: Try our Seminar/Events search on the homepage.) Please address
%   also the relation and differences of your proposed seminar to (major)
%   conferences in the same area.

\section{Introduction}

From their earliest days, computers have successfully been used in mathematics
to perform complicated or tedious calculations more reliably, or at all, to make
tables, to prove theorems (famously the four colour theorem), to explore new
theories.

With computers and open source software becoming widely, and cheaply, available
to everyone, the last decades have seen the emergence of open-source tools to
conduct research; the spectrum ranges from special purpose Excel worksheets,
to more sophisticated libraries such as \MPIR or \Linbox, over topical systems
such as \Singular, to general purpose systems that feature a complete
programming language and environment such as \GAP or to a much larger extent
\Sage.

Addionally there is a wealth of databases of objects, Atlas, small groups,
transitive groups, findstat, regular graphs,...
\ednote{markus: I might be focusing this too much on pure maths/combinatorics?}

There is also undeniable value in these software packages for teaching and
collaborative work.

A big problem with the multitude of systems with different communities,
developers, and focuses, is that they do not compose well. To have any hope of
using a mathematics software package inside one's own system one has to
understand the conventions and internals of the library sufficiently to write a
bespoke interface, translate data and representations of objects semantically
correctly.
This is tedious, time consuming and error-prone.

In an ideal world there would be a single, well-defined, standard that is used
to efficiently communicate data between systems.
uch a standard would have to be universal enough to cover enough fields,
easy to use to not impose an implementation burden on developers,
and efficient enough to not slow down computations significantly.

Since the world rarely is ideal, and one cannot expect every developer of
mathematics software to change it to the needs of this standard, we need to work
towards composability.

A partial success is \Sage, a free general purpose open-source mathematics
software system licensed under the GPL. Its mission is to create a
viable free open source alternative to Magma, Maple, Mathematica and
Matlab.
\Sage has been developed since 2005 by a growing worldwide community of
about 150 researchers and teachers. It builds on top of many existing
open-source packages, including NumPy, SciPy, matplotlib, Sympy,
Maxima, and all other packages mentioned before.
All packages are accessible from a Python-based library, which itself
contains many unique mathematical features building on top of the included packages.
This makes \Sage a popular choice for research and education purposes.

We propose to establish the Math-in-the-Middle approach as described in
\ednote{CICM paper?} based on MMT as the base for exchanging mathematical
objects/meaning/etc; a universal API for mathematics.


\section{Description}

Computer mathematics software is very diverse: in the past every special
interest community developed their software only considering their own
needs, with custom APIs and environments. Some projects are developed in general
purpose programming languages, such as C, Java, or Python, others develop their
own domain specific languages, or custom languages (like GAP or GP).

Contemporary research in computational mathematics often needs access to
multiple diverse specialist libraries (example; groups, number theory, linear algebra...) 

The upshot is that researchers build software by tacking it together with duct
tape and patches, or duplicate effort developing specialist solutions for their
current problems, which comes at cost of fragility, complexity, and doubtful correctness.

But a general framework that connects exists: its formal mathematics, formalised
in the language of logic.

Rabe/Kohlhase et al developed a very general framework (MMT) that can serve as a
translation layer between different logical frameworks.
(need a description of the mmt approach?)

The vision is to establish this Math-in-the-middle approach as the API for as
many computer mathematics software, without forcing these packages to change
their approach.

Working on WP6 of OpenDreamKit we realised more and more that it is necessary to
communicate and work together in one place

get representatives from fields to compose

The seminar will consist of few talks that introduce the core concepts of MMT,
current state of exports of Sage, GAP, Lmfdb MMT interfaces

Lower the entry barrier for using MitM

\section{Questions, Issues, Objectives, Results}

Conference, seminars, etc. often focus on the exchange of ideas. It is
of course one of the goals of our proposed semainar, in particular
brushing a precise state-of-the-art of the above topics. 
However our goals go beyond that: we wold like to develop a common
objective of universality that can unify the communities. In the long run, this requires designing uniform conceptualizations, interchange formats, and interoperability layers.

Therefore, \textbf{the goal of this seminar is to systematically identify the
current obstacles to universality, to collect requirements for universality, and
to sketch out future solutions}.
\ednote{markus: I think this should be mentioned much earlier, in the introduction or abstract}

The goal of the seminar is to 
\begin{itemize}
\item Bring together developers and users of (open source) mathematics software,
  and logicians and knowledge representation experts.
\item Get an overview of the current trends and developments in open source
  mathematics software.
\item Promote MMT/Math-in-the-Middle
\item Train mathematicians in the art of MMT, and get an idea of how MMT will be
  used by developers of domain specific software.
\item Share perspectives and best practices, build a joint vision, and
  seek venues for tighter collaboration.
\item Encourage participants to get involved in the standardisation process and
  to provide MMT interfaces to their software.
\end{itemize}

Some of the upcoming major challenges are:
\begin{itemize}
\item Lower the entry barrier, in particular via \textbf{unified user
    interfaces}, and \textbf{Virtual Research Environments} that
  groups of users can setup to collaborate on data, software,
  computations, or knowledge;
\item Further enable \textbf{computations involving multiple systems},
  as transparently as possible;
\item Keep the development efforts manageable as the size and
  complexity of software systems increase;
\item Train a new generation of users and developers.
\end{itemize}

A key step is to strengthen collaborations \emph{between} the various
communities, in order to:
\begin{itemize}
\item Seek for opportunities for collaboration or outsourcing of
  components to save on development efforts;
\item Share expertise and best practices;
\item Improve cross-systems development workflows.
\end{itemize}

\paragraph{Research Questions}
% checking dependency theories
% identification of assumptions: extensionnality, choice, proof
% irrelevance etc
% tools for packaging, maintenaning
Participants will be asked to give short talks that specifically address the following research questions from the perspective of their field:

\begin{itemize}
\item Why are current systems not more interoperable? What design changes are necessary to increase interoperability in the future?
\item What are the current approaches towards interoperability? How successful or promising are they?
\item How can correctness be guaranteed in a distributed setting?
 Should there be a single universal checker (which would be hard to agree on) or many decentral ones (which may preclude interoperability)?
\item How can we design interchange languages that naturally subsume
  existing (and future!) formal systems?
\item Should a logical framework permit the definition of any logical system?
Or do the logics currently implemented have points in common that could
be hard wired into the framework itself?
\item How reasonnable is it to propose a single universal proof format?
Or do we need different formats for different families of
proof systems and a partial interoperability between the formats?
 How should a proof format be evaluated (generality, conciseness,
efficiency of proof-checking, ...)?
\item How should universal proof library be exchanged? Is Web technology
sufficient or do we need specific tools to organize data bases of
proofs?
\item How can we practically and reliably relate individual systems with their representation in an interchange format or a logical framework?
How can two systems agree on the meaning of an exchanged theorem and thus trust each other?
\end{itemize}

\paragraph{Impact on the Research Community}
By challenging participants to address research questions concerning
universality, we do not only raise awareness of the importance of these issues.
We also help identify the key steps towards \emph{proving in the large} and  \emph{universality} of proofs.
This will allow the development of a common objective and framework for interoperable and reusable proof development that is crucial for realizing the full potential of formal mechanizations.

This seminar with the associated Dagstuhl proceedings will provide an overview of the problem, the state of the art of current solutions and the active researchers pursuing them, and the most promising ideas for future solutions.
It will collect and strengthen the small, often-disparate communities that currently work towards universality, e.g., in the very different PxTP (Proof eXchange for Theorem Proving) and LFMTP (dedicated to logical frameworks and meta-languages) workshops.

The seminar will not only allow for cross-fertilization between
 \begin{compactitem}
  \item research on logical frameworks, proof formats, logics, proof engineering, mathematics formalization, and program verification,
  \item foundational research on these topics and application or system-oriented approaches.
 \end{compactitem}
It will also structure and streamline future collaboration, e.g., by kicking off new workshops or large international grant proposals.




% % % % % % % % % % % % % % % % % %
\subsection{Relation to Previous Dagstuhl Seminars}
\input{previous}

% % % % % % % % % % % % % % % % % %
\section{List of Potential Participants}

We have identified \ednote{add number} relevant researchers with a particular focus on
\begin{compactitem}
  \item including leading experts from all involved communities,
  \item bringing together researchers with particular interest in universality and interoperability.
\end{compactitem}
\medskip

\ednote{@all: revise areas and subareas depending on what people we want; the topics listed here will be referenced in the spreadsheet containing the suggested participants}

\noindent
We carefully selected participants to cover the following areas:
\begin{compactitem}
\item computation systems
  \begin{compactitem}
    \item symbolic computation
    \item exact computation
    \item numerical computation
  \end{compactitem}
\item related systems
  \begin{compactitem}
    \item user interfaces
    \item databases
    \item knowledge bases
  \end{compactitem}
\item applications
  \begin{compactitem}
    \item mathematical databases
    \item scientific computation
    \item industrial applications
  \end{compactitem}
\item system integration
  \begin{compactitem}
    \item integration frameworks
    \item individual system connections
  \end{compactitem}
\end{compactitem}
\medskip

\noindent
Moreover, the participants include\ednote{add numbers at the end}
\begin{compactitem}
\item XXX from Germany, XXX from France, XXX from UK, XXX from the rest of Europe, XXX from North America
\item XXX junior, XXX female (one of them co-organizer), and XXX industrial researchers.
\end{compactitem}

% % % % % % % % % % % % % % % % % %
\section{Information on the Organizers}
\ednote{add a 0.5-1 page research CV of yourself: overview of an organizer's academic career, especially points out community services and recognitions, list the five most relevant papers}

\subsection{Brief presentation of the organizers}
Florian Rabe has developed the MMT framework and the LF module system and is the main contributor of the LATIN atlas. He has extensive expertise in individual deduction systems including major case studies regarding Mizar, HOL Light, PVS, and TPTP.

Viviane Pons is a researcher in Combinatorics, leader of the dissemination work package of the OpenDreamKit H2020 project. She has been involved in the development of SageMath. She has been organizing Sage Days and given many Sage interventions and tutorial.

 \newpage
  \subsection{Florian Rabe - CV}
    \newcommand{\tb}{\hspace*{1cm}}

\begin{tabularx}{\textwidth}{lX}
\textbf{Name}           & Dr. habil. Florian Rabe, born 1979-09-28 \\
\textbf{Diploma}        & Computer science 2004, Universitity of Karlsruhe (distinction) \\
\textbf{PhD}            & Computer sciecne 2008, Jacobs University Bremen (distinction) \\
\textbf{Habilitation}   & Computer sciecne 2014, Jacobs U. \\
\textbf{Employment}     & 2008--2014, post-doctoral fellow, Jacobs U. \\
                        & 2014-- DFG-Eigene Stelle, Jacobs U.\\
%\textbf{Gastaufenthalte}        & \\
% \tb 2006 (12 Monate)   & Carnegie Mellon University, Pittsburgh, USA, bei Prof. Frank Pfenning (DAAD-Stipendium)\\
% \tb Jan. 2009 (1 Monat)& IT University Kopenhagen, D\"anemark, bei Prof. Carsten Sch\"urmann (eingeladen) \\
% \tb Jun. 2010 (1 Monat)& IT University Kopenhagen, D\"anemark, bei Prof. Carsten Sch\"urmann (eingeladen)\\
% \tb Jan. 2011          & MacMaster University, Hamilton, Ontario, Kanada, bei Prof. William Farmer, Prof. Jacques Carette\\
% \tb Jun. 2013          & Universit\"at Z\"urich, Schweiz, bei Prof. Paul-Olivier Dehaye (eingeladen) \\
% \tb Feb. 2014          & Universit\"at Innsbruck, \"Osterreich, bei Dr. Cezary Kaliszyk (eingeladen) \\
% \tb Sep. 2014          & Chalmers University of Technology, G\"oteborg, Schweden, bei Dr. Cezar Ionescu (eingeladen) \\
% \tb M\"arz-April       & SRI International, Menlo Park, California, US, bei Dr. Natarajan Shankar, \\
% \tb \tb (1.5 Monate)   &  \tb und Kestrel Institute, Palo Alto, California, US, bei Dr. Cordell Green \\
\multicolumn{2}{l}{\textbf{Awards and Scholarships}}  \\
\tb  2005               & Best diploma thesis, Computer science faculty \\
\tb  2006               & PhD scholarship (1 year), DAAD\\
\tb  2006               & Winner Modal Logic \$100 Challenge \\
\tb  2007--2008         & PhD scholarship, German Merit Foundation\\
\tb  2010               & Best Paper Award, MKM conference\\
\tb  2015               & Contest Winner ``The Future of Logic'', UniLog Congress \\
\multicolumn{2}{l}{\textbf{Membership in Academic Self-governance Committees}} \\
\tb 2008 -- 2010        & Staff Council (Jacobs U.)\\ 
\tb 2010 - 2012         & Provost search committee (Jacobs U.) \\
\tb 2011 - 2012         & Constitution committee (Jacobs U.) \\
\tb 2010 -- 2013        & Board of trustees of MKM interest group\\
\tb 2012 --             & Steering committee of CICM conference\\
\textbf{Student Advising}       & 14 BSc., 6 MSc., 3 PhD. (some in progress)  \\
\textbf{Organization}   & 2 conferences, 4 workshops \\
\textbf{PC Membership} & 8 conferences (2 as track chair), 11 workshops (4 as chair) \\
%\textbf{Lehre}          & 2 Undergraduate-Vorlesungen, 4 Undergraduate-Labs, \\
%                        & 3 Graduate-Vorlesungen, 6 Graduate-Seminare und -Labs \\
\multicolumn{2}{l}{\textbf{Third Party Funding}} \\
\tb 2009--2012             & LATIN (DFG), de-facto PI \\
\tb 2014--2017             & OAF (DFG), lead PI \\
\tb 2015--2019             & OpenDreamKit (EU Horizon 2020), PI \\
\end{tabularx}
\medskip

%\renewcommand{\refname}{Important publications}
%\providecommand{\etalchar}[1]{$^{#1}$}
\renewcommand{\bibitem}[2][]{\item}

\noindent
\textbf{$5$ Important Publications}
\begin{compactitem}%{GMd{\etalchar{+}}07}
\bibitem[KR15]{KR:qed:14}
M.~Kohlhase and F.~Rabe.
\newblock {QED Reloaded: Towards a Pluralistic Formal Library of Mathematical
  Knowledge}.
\newblock {\em Journal of Formalized Reasoning}, 2015.
\newblock accepted pending minor revisions; see
  \url{http://kwarc.info/frabe/Research/KR_qed_14.pdf}.

\bibitem[Rab15b]{rabe:future:15}
F.~Rabe.
\newblock {The Future of Logic: Foundation-Independence}.
\newblock {\em Logica Universalis}, 2015.
\newblock Winner of the Contest ``The Future of Logic'' at the World Congress
  on Universal Logic; to appear; see
  \url{http://kwarc.info/frabe/Research/rabe_future_15.pdf}.

\bibitem[Rab14]{rabe:howto:14}
F.~Rabe.
\newblock {How to Identify, Translate, and Combine Logics?}
\newblock {\em Journal of Logic and Computation}, 2014.
\newblock doi:10.1093/logcom/exu079.

\bibitem[RK13]{RK:mmt:10}
F.~Rabe and M.~Kohlhase.
\newblock {A Scalable Module System}.
\newblock {\em Information and Computation}, 230(1):1--54, 2013.

\bibitem[HR11]{HR:folsound:10}
F.~Horozal and F.~Rabe.
\newblock {Representing Model Theory in a Type-Theoretical Logical Framework}.
\newblock {\em Theoretical Computer Science}, 412(37):4919--4945, 2011.
\end{compactitem}





    
  \newpage
  \subsection{Viviane Pons - CV}
    \begin{tabularx}{\textwidth}{lX}
\textbf{Name}           & Dr. Viviane Pons, born 1985-02-05 \\
\textbf{Bachelor Degree}        & Computer science and Mathematics 2006, Univ. Paris-Est (distinction) \\
\textbf{Master Degree}        & Computer science 2010, Univ. Paris-Est (distinction) \\
\textbf{PhD}            & Computer science 2013, Univ. Paris-Est \\
\textbf{Employment}     & 2013--2014, post-doctoral researcher, Univ. of Vienna (Austria) \\
                        & 2014--, Ma\"itre de conf\'erences en Informatique, Univ. Paris-Sud\\
%\textbf{Gastaufenthalte}        & \\
% \tb 2006 (12 Monate)   & Carnegie Mellon University, Pittsburgh, USA, bei Prof. Frank Pfenning (DAAD-Stipendium)\\
% \tb Jan. 2009 (1 Monat)& IT University Kopenhagen, D\"anemark, bei Prof. Carsten Sch\"urmann (eingeladen) \\
% \tb Jun. 2010 (1 Monat)& IT University Kopenhagen, D\"anemark, bei Prof. Carsten Sch\"urmann (eingeladen)\\
% \tb Jan. 2011          & MacMaster University, Hamilton, Ontario, Kanada, bei Prof. William Farmer, Prof. Jacques Carette\\
% \tb Jun. 2013          & Universit\"at Z\"urich, Schweiz, bei Prof. Paul-Olivier Dehaye (eingeladen) \\
% \tb Feb. 2014          & Universit\"at Innsbruck, \"Osterreich, bei Dr. Cezary Kaliszyk (eingeladen) \\
% \tb Sep. 2014          & Chalmers University of Technology, G\"oteborg, Schweden, bei Dr. Cezar Ionescu (eingeladen) \\
% \tb M\"arz-April       & SRI International, Menlo Park, California, US, bei Dr. Natarajan Shankar, \\
% \tb \tb (1.5 Monate)   &  \tb und Kestrel Institute, Palo Alto, California, US, bei Dr. Cordell Green \\
\multicolumn{2}{l}{\textbf{Projects}} \\
\tb 2015 -- 2019        & OpenDreamKit (EU Horizon 2020)\\ 
\tb         & Site leader for Paris-Sud  \\
\tb         & Community building / Dissemination Work package leader (WP2) \\
\textbf{Organization}   & 4 workshops \\
\multicolumn{2}{l}{\textbf{Invited speaker / lecturer}} \\
\tb 2015 & Combinatorics and Sage at  EAUMP Summer school on experimental mathematics (Uganda) \\
\tb 2016 & Combinatorics Research School at ENS Lyon (France) \\
\tb 2016 & ECCO 2016 -- Sage Tutorial (Colombia) \\
\tb 2016 & Codima School 2016 -- Sage and SageMathCloud Tutorial (Scotland) \\ 
\multicolumn{2}{l}{\textbf{Other outreaching activities}} \\
\tb 2014 -- 2015 & Experimental mathematics using Sage: talks at SciPy2015 PyConFr2014 and PyCon2015 \\
\tb 2016 -- & Co-organizer of the PyLadies Paris Chapter \\
\textbf{PC Membership} & FPSAC 2017 (conference) \\
\end{tabularx}
\medskip

%\renewcommand{\refname}{Important publications}
%\providecommand{\etalchar}[1]{$^{#1}$}
\renewcommand{\bibitem}[2][]{\item}

\noindent
\textbf{Important Publications}
\begin{compactitem}%{GMd{\etalchar{+}}07}
\bibitem[CP15]{CP:tamari:15}
G.~Ch\^atel and V.~Pons.
\newblock { Counting smaller elements in the tamari and m-tamari lattices}.
\newblock {\em Journal of Combinatorial Theory, Series A}, 2015.
\newblock doi:10.1016/j.jcta.2015.03.004.

\bibitem[Pon13]{Pon:groth:13}
V.~Pons.
\newblock {Interval structure of the Pieri formula for Grothendieck polynomials}.
\newblock {\em International Journal of Algebra and Computation}, 2013.
\newblock doi:10.1142/S0218196713500045. 

\bibitem[Pon11]{Pon:poly:11}
V.~Pons.
\newblock {Multivariate polynomials in Sage}
\newblock {\em Séminaire Lotharingien de Combinatoire}, 2011.
\end{compactitem}






\end{document}