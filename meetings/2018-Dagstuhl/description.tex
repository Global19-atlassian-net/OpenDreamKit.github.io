
% 2.1 Description of the seminar (3-5 pages, in English):
% - Brief, general introduction to the topic
% - In-depth description of the topic
% - Questions and issues addressed by the seminar; objectives and results
%   expected to be produced by the seminar
% - As applicable: relationship to previous seminars or how the proposed seminar
%   differs from similar seminars. Which new developments and issues are to be
%   addressed? (Tip: Try our Seminar/Events search on the homepage.) Please address
%   also the relation and differences of your proposed seminar to (major)
%   conferences in the same area.

\section{Introduction}

From their earliest days, computers have successfully been used in mathematics
to perform complicated or tedious calculations more reliably, or at all, to make
tables, to prove theorems (famously the four colour theorem), to explore new
theories.

With computers and open source software becoming widely, and cheaply, available
to everyone, the last decades have seen the emergence of open-source software in
research communities; the spectrum ranges from special purpose Excel worksheets,
to more sophisticated libraries such as \MPIR (for high performance integer
arithmetic) or \Linbox (linear algebra), to topical systems
such as \Pari in number theory, \Singular in algebraic geometry, or \GAP in group
theory.
Some packages are written and embedded in popular general purpose programming
languages, wheras some systems feature their own programming language.

Addionally, there is a wealth of databases of mathematical objects: the online
Atlas of Finite Simple Groups, the Small Groups Library, the L-functions and
Modular Forms Database, FindStat, the Regular Graphs library, the Online
Encyclopedia of Integer Sequences, and many more.
\ednote{markus: I might be focusing this too much on pure maths/combinatorics?}

One major problem posed by the multitude of systems with different communities,
developers, and focuses, is that they do not compose well. The current state of
the art in interfacing between systems is to understand the conventions and
internals of the systems involved sufficiently to write a bespoke interface, and
to translate data and representations of objects semantically correctly.

As an example, the group-centric computer algebra system \GAP includes custom
interfaces to \Singular, the Graph Isomorphism Solvers \Nauty and \Bliss,
separately with different representations of graphs, the advanced coset
enumerator \ACE, Holt's automatic groups programs \KBMAG, and more.

\Sage, a free general purpose open-source mathematics software with a mission to
provide an alternative to MAGMA, Maple, Mathematica, and Matlab, includes \GAP
as a subsystem. The interface between \Sage and \GAP consists of hand-written
translation routines, which for example have to interface \Sage's integers with
\GAP's integers, and \Sage's permutations with \GAP's.

This method of interfacing systems is tedious, time consuming, and error-prone.

Ideally there would be a single, well-defined, standard that defiens how to
efficiently and correctly communicate data between systems.
Such a standard would have to be universal enough to cover most of
computational mathematics, easy to use, and efficient.

(maybe itemize? what about composability/transitivity?
\begin{itemize}
\item \textbf{Universality}
\item \textbf{Usability}
\item \textbf{Efficiency}
\end{itemize}
)

One cannot expect every developer of mathematics software to change it to
the needs of this standard, we need to work towards
composability.\ednote{Markus: This needs more substance}

\section{The Math-In-The-Middle Approach}

To address the issues introduced in the preceding section, we propose to
establish the Math-in-the-Middle based on OMDoc/MMT approach as described
in \ref{cicm2016}.

The idea is to flexibly formalise mathematical concepts in a very general
logical framework, and to encourage authors to describe their systems in this
framework, ultimately establishing a common \emph{meaning space} in which data
can be automatically, efficiently, and semantically correctly translated between
systems.
Parts of the process of description and discovery can in many cases be
automated, as is witnessed by our work on \GAP, \Sage, and \Lmfdb.

The vision is to establish this Math-in-the-middle approach as the universal API
for as many computer mathematics software packages as possible, without forcing
these packages to change their software significantly.

Working on WP6 of OpenDreamKit we realised more and more that it is necessary to
communicate and work together in one place

The seminar will consist of few talks that introduce and showcase our initial
successes working on WP6 of ODK involving Sage, GAP, LMFDB, the core concepts
of MMT, and many workgroup sessions to learn from each other about developers
needs and about barriers to entry and use. 

\section{Questions, Issues, Objectives, Results}

\textbf{The central goal of this seminar is to systematically identify the
current obstacles to universality, usability, and efficiency, and
to sketch out future solutions}.

To achieve this goal, we will 
\begin{itemize}
\item Bring together developers and users of (open source) mathematics software,
  logicians, and knowledge representation experts.
\item Get an overview of the current trends and developments in open source
  mathematics software.
\item Promote MMT/Math-in-the-Middle
\item Train mathematicians in the art of MMT, and get an idea of how MMT will be
  used by developers of domain specific software.
\item Share perspectives and best practices, build a joint vision, and
  seek venues for tighter collaboration.
\item Encourage participants to get involved in the standardisation process and
  to provide MMT interfaces to their software.
\end{itemize}

Some of the upcoming major challenges are:
\begin{itemize}
\item Enabling \textbf{computations involving multiple systems},
  as transparently as possible;
\item Keeping the development efforts manageable as the size and
  complexity of software systems increase;
\item Training a new generation of users and developers.
\item Combining the above to remove usability barriers, in particular via
  \textbf{unified user interfaces}, and \textbf{Virtual Research Environments}
  that groups of users can setup to collaborate on data, software, computations,
  or knowledge;
\end{itemize}

A key step is to strengthen collaborations \emph{between} communities, in order
to:
\begin{itemize}
\item Seek for opportunities for collaboration or outsourcing of
  components to save on development efforts;
\item Share expertise and best practices;
\item Improve cross-systems development workflows.
\end{itemize}

\paragraph{Research Questions}
% checking dependency theories
% identification of assumptions: extensionnality, choice, proof
% irrelevance etc
% tools for packaging, maintenaning
Participants will be asked to give short talks that specifically address the following research questions from the perspective of their field:

\begin{itemize}
\item Why are current systems not more interoperable? What design changes are necessary to increase interoperability in the future?
\item What are the current approaches towards interoperability? How successful or promising are they?
\item How can correctness be guaranteed in a distributed setting?
 Should there be a single universal checker (which would be hard to agree on) or many decentral ones (which may preclude interoperability)?
\item How can we design interchange languages that naturally subsume
  existing (and future!) formal systems?
\item Should a logical framework permit the definition of any logical system?
Or do the logics currently implemented have points in common that could
be hard wired into the framework itself?
\item How reasonnable is it to propose a single universal proof format?
Or do we need different formats for different families of
proof systems and a partial interoperability between the formats?
 How should a proof format be evaluated (generality, conciseness,
efficiency of proof-checking, ...)?
\item How should universal proof library be exchanged? Is Web technology
sufficient or do we need specific tools to organize data bases of
proofs?
\item How can we practically and reliably relate individual systems with their representation in an interchange format or a logical framework?
How can two systems agree on the meaning of an exchanged theorem and thus trust each other?
\end{itemize}

\paragraph{Impact on the Research Community}
By challenging participants to address research questions concerning
universality, we do not only raise awareness of the importance of these issues.
We also help identify the key steps towards \emph{proving in the large} and
\emph{universality} of proofs. This will allow the development of a common
objective and framework for interoperable and reusable proof development that is
crucial for realizing the full potential of formal mechanizations.

This seminar with the associated Dagstuhl proceedings will provide an overview
of the problem, the state of the art of current solutions and the active
researchers pursuing them, and the most promising ideas for future solutions. It
will collect and strengthen the small, often-disparate communities that
currently work towards universality, e.g., in the very different PxTP (Proof
eXchange for Theorem Proving) and LFMTP (dedicated to logical frameworks and
meta-languages) workshops.

The seminar will not only allow for cross-fertilization between
 \begin{compactitem}
  \item research on logical frameworks, proof formats, logics, proof
engineering, mathematics formalization, and program verification,
  \item foundational research on these topics and application or system-oriented
approaches.
 \end{compactitem} It will also structure and streamline future collaboration,
e.g., by kicking off new workshops or large international grant proposals.

