\documentclass[12pt]{amsbook}
\usepackage[utf8]{inputenc}
\usepackage{ae,aecompl,aeguill}	% pour utiliser << et >>
\usepackage[francais]{babel}
\frenchbsetup{og=«, fg=»}
\usepackage{times}
\usepackage[babel=true,kerning=true]{microtype}
\usepackage{hyperref}

\newcommand{\scienceproject}{\mbox{\textsc{SCIEnce}}\xspace}

\newcommand{\OOMMFNB}{OOMMF-NB\xspace}
\newcommand{\VRE}{VRE\xspace}
\newcommand{\VREs}{VRE\xspace}

\newcommand{\software}[1]{\textsc{#1}\xspace}

\newcommand{\GAP}{\software{GAP}}
\newcommand{\HPCGAP}{\software{HPC-GAP}}
\newcommand{\libGAP}{\software{libGAP}}
\newcommand{\Singular}{\software{Singular}}
\newcommand{\Sage}{\software{Sage}}
\newcommand{\SageCombinat}{\software{Sage-Combinat}}
\newcommand{\MuPADCombinat}{\software{MuPAD-Combinat}}
\newcommand{\Sphinx}{\software{Sphinx}}
\newcommand{\SCSCP}{\software{SCSCP}}
\newcommand{\Python}{\software{Python}}
\newcommand{\IPython}{\software{IPython}}
\newcommand{\Jupyter}{\software{Jupyter}}
\newcommand{\Cython}{\software{Cython}}
\newcommand{\Pythran}{\software{Pythran}}
\newcommand{\Numpy}{\software{Numpy}}
\newcommand{\Pari}{\software{PARI}}
\newcommand{\PariGP}{\software{PARI/GP}}
\newcommand{\libpari}{\software{libpari}}
\newcommand{\GP}{\software{GP}}
\newcommand{\GPtoC}{\software{GP2C}}
\newcommand{\Linbox}{\software{LinBox}}
\newcommand{\LMFDB}{\software{LMFDB}}
\newcommand{\OpenEdX}{\software{OpenEdX}}
\newcommand{\Linux}{\software{Linux}}
\newcommand{\LATEX}{\software{\LaTeX}}
\newcommand{\SMC}{\software{SageMathCloud}}
\newcommand{\Simulagora}{\software{Simulagora}}
\newcommand{\KANT}{\software{KANT}}
\newcommand{\Magma}{\software{Magma}}
\newcommand{\Mathematica}{\software{Mathematica}}
\newcommand{\Maple}{\software{Maple}}
\newcommand{\Matlab}{\software{Matlab}}
\newcommand{\MuPAD}{\software{MuPAD}}
\newcommand{\MPIR}{\software{MPIR}}
\newcommand{\Arxiv}{\software{arXiv}}
\newcommand{\Givaro}{\software{Givaro}}
\newcommand{\fflas}{\software{fflas}}
\newcommand{\MathHub}{\software{MathHub}}
\newcommand{\xcas}{\software{Giac/Xcas}}
\newcommand{\Mathemagix}{\software{Mathemagix}}
\newcommand{\FindStat}{\software{FindStat}}
\newcommand{\OSCAR}{\software{OSCAR}}
\newcommand{\Nauty}{\software{Nauty}}
\newcommand{\Bliss}{\software{Bliss}}
\newcommand{\ACE}{\software{ACE}}
\newcommand{\KBMAG}{\software{KBMAG}}

\newcommand\DKS{\ensuremath{\mathcal{DKS}}\xspace}
\newcommand\OEIS{\ensuremath{\mathcal{OEIS}}\xspace}

\newcommand{\ODK}{\href{http://opendreamkit.org}{OpenDreamKit}\xspace}

%%% Local Variables: 
%%% mode: latex
%%% TeX-master: "proposal"
%%% End: 


\begin{document}

\title{Sage Days}

\section{Scientific context}


From their earliest days, computers have been used in pure
mathematics, either to make tables, to prove theorems (famously the
four colour theorem) or, as with the astronomer's telescope, to
explore new theories. Computer-aided experiments, and the use of
databases relying on computer calculations such as the Small Groups
Library in GAP, the Modular Atlas in group and representation theory,
or the $L$-functions and Modular Forms Database (\LMFDB, see later),
are part of the standard toolbox of the pure mathematician, and
certain areas of mathematics completely depend on it. Computers are
also increasingly used to support collaborative work and education.

The last decades witnessed the emergence of a wide ecosystem of
open-source tools to support research in pure mathematics. This ranges
from specialized to general purpose computational tools such as \GAP,
\PariGP, \Linbox, \MPIR, \Sage, or \Singular, via online databases
like the \LMFDB and does not count online services like Wikipedia,
\Arxiv, or MathOverflow. A great opportunity is the rapid emergence of
key technologies, and in particular the \Jupyter (previously \IPython)
platform for interactive and exploratory computing which targets all
areas of science. This has proven the viability and power of
collaborative open-source development models, by users and for users,
even for delivering general purpose systems targeting a large public
(researchers, teachers, engineers, amateurs, \ldots).

An exemplary success in the last decade is that of \Sage,
a free general purpose open-source mathematics
software system licensed under the GPL whose mission is to create a
viable free open source alternative to Magma, Maple, Mathematica and
Matlab. It's developed since 2005 by a growing worldwide community of
about 150 researchers and teachers. It builds on top of many existing
open-source packages, including NumPy, SciPy, matplotlib, Sympy,
Maxima, and the aforementioned ones, all accessible from a
Python-based library containing itself many unique mathematical
features. \Sage can be used as well as batch program, through the
\Jupyter interactive computing interface (command-line or graphical),
or \href{cloud.sagemath.org}{on the cloud}.

Thanks to this, \Sage is used regularly in universities, both for
research and education; an indicator of its success in the French
system is that, since 2014, \Sage is in the short list of official
software for the oral examinations of the Agrégation de Mathématiques
(the nation-wide recruiting competition for high school teachers), and
that since 2015 this list contains only open source software.

\section{Toward multi-system and virtual research environments}

...

\section{The \Sage days tradition}

The success of any research software or service is strongly linked to
its ability to attract and retain a large number of users. The
different communities (\Sage, \GAP, \Pari, \Singular, \Jupyter, ...)
have each developed sustainable networks. For example, \Sage has
accumulated thousands of users in under 10 years.

This has been achieved thanks to a very strong community building
philosophy, especially through the organization of
“\href{https://wiki.sagemath.org/Workshops}{Sage-Days}” all over the
world. The first Sage-Days was held in 2006 with 10 participants; to
date there has been at least 77 of them, including about ten in
2015. Those workshops are typically week-long, and cover a wide
variety of topics, ranging from focused workshops gathering a dozen
developers for a coding-sprints on specific mathematical or technical
features to training workshops with more than 80 people. To cite just
a few of the recent or upcoming ones:

\begin{itemize}
\item Sage Education Days 6 (June 16-18, 2014, University of
  Washington, Seattle)
\item Sage Days 69 (September 4-9, 2015, San Diego): Women in Sage 6
\item Sage Days 73 (May 04-07, 2016, Oaxaca (Mexico)): Translation surfaces
\item Sage Days 77 (April 04-08, 2016. Cernay, Paris area, France):
  Packaging, portability, documentation tools
\item Sage Days 79 (November 21-25, Jerusalem, Israel): Combinatorics
\end{itemize}

All those workshops have in common to be highly hands on, with lots of
room for tutorials and collaborative work, and a strong dynamic
(project sessions, status reports, ...) to get everyone involved.

\subsection{Previous \Sage Days at CIRM}

One particularly successful workshop,
\href{https://www.lirmm.fr/arith/wiki/MathInfo2010/SageDays}{\Sage
  Days 20}, was organized at CIRM in 2010, at the occasion of the
thematic month
\href{https://www.lirmm.fr/arith/wiki/MathInfo2010/}{Math-Info}. This
workshop had a double focus on training and research. It was one of
the largest \Sage Days, with about 80 participants, many of which
became at this occasion regular \Sage users, if not contributors. One
of the many outcome was a strong initial impetus to what was to become
the first book about \Sage: \href{Calcul Mathématique avec
  Sage}{http://sagebook.gforge.inria.fr/}, in French and under an open
source license. Also a lot of training material was written during the
workshop and reused extensively in followup workshops; a large chunk
of this material was integrated in the \Sage official documentation in
the form of thematic tutorials.

The commodities at CIRM turned out just right for the purpose, and
since then the proposers have been desiring to organize another one.

\subsection{Experience of the proposers}

All the proposers have a strong experience with \Sage Days. For
example, Nicolas Thiéry attended 20+ \Sage Days, was main organizer or
coorganizer of more than a dozen of them, and invited speaker in many
of the others, including that at CIRM.

\section{A cofunding opportunity: OpenDreamKit}

\href{http://opendreamkit.org}{OpenDreamKit} is a Horizon 2020
European Research Infrastructure project (\#676541) funded under the
EINFRA-9 call~\cite{EINFRA-9}, that will run for four years starting
from September 2015. It will provide substantial funding to the open
source computational mathematics ecosystem, and in particular popular
tools such as LinBox, MPIR, SageMath, GAP, Pari/GP, LMFDB, Singular,
MathHub, and the IPython/Jupyter interactive computing environment.

``From this ecosystem, \ODK will deliver a flexible toolkit enabling
research groups to set up Virtual Research Environments, customised to
meet the varied needs of research projects in pure mathematics and
applications, and supporting the full research life-cycle from
exploration, through proof and publication, to archival and sharing of
data and code.''

The \ODK consortium consists of core European developers of the
aforementioned systems for pure mathematics, and reaching toward the
numerical community, and in particular the \Jupyter community, to work
together on joint needs.

\textbf{Dissemination and training is at the heart of \ODK}. From the
beginning of the design of the proposal, back in 2014 (during ALEA
2014 at CIRM!), it was planned to fund yearly large meetings to reach
toward the larger community, disseminate the outcome, and train new
users and developers. Besides CIRM in 2018, we are aiming for meetings
in Dagstuhl, and ICMS (Edimburgh).

% Yet some critical long term investments, in particular on the
% technical side, are in order to boost the productivity and lower the
% entry barrier:

\section{Joint dynamics with JNCF 2018}

\section{Goals}

\subsection{Community building}

Bringing together the community of users and developpers of the
ecosystem of (open source) (pure) mathematics software (80 people)

Give newcomers as well as specialists an overview of this ecosystem:
what are the existing software, what kind of problems can be solved
with them, how they are developped, and by whom. And train them hard
on using this ecosystem to solve their own problems!

Share perspectives and best practices. Build a joint vision and seek
venues for tighter cooperation.

Encourage people to get involved, in particular young and women, in
particular by exposing them with role models.

\subsection{Training}

\subsection{}

\subsection{Public and prerequisites}



\section{Program}

The program will consists of keynote talks (one or two per day), many
hands on tutorials, and a lot of free time for collaborative work and
coding sprints.

\subsection{Keynote talks}

We are aiming at 5-6 one-hour keynote talks. The goal is to deliver a
variety of perspectives on the ecosystem of open source software for
mathematics. At this point we have five confirmed speakers:

\begin{description}
\item[Historical perspective on contributions of researchers and
  teachers to (open source) mathematical software]
  Marie-Françoise Roy
\item[Design of general purpose mathematical software: Mathemagix (and TeXmacs)]
  Joris VanDerHoeven
\item[Collaborative software development in a large system (GAP)]
  Max Horn
\item[Scientific and interactive computing, User interfaces (e.g. Jupyter) and community]
  Fernando Perez
\item[Experimental mathematics]
  Anne Schilling
\end{description}

We are aiming at complementing those with speakers presenting the
developers perspective from a large system like GAP and a specialized
library. We have a short list of speakers for those, but decided to
postpone the decision and invitations, in order to keep some
flexibility to accommodate for potential evolutions in the coming
months as we reach toward neighbor communities (e.g. proof systems).

\subsection{Hands-on tutorials}

We will reserve 6-8 slots for hands-on tutorial on the various
software of the ecosystem. Each tutorial will be delivered by experts
(typically one leader and several helpers). After a brief overview,
the participants will be guided through a collection of exercise
worksheets, designed to accommodate various levels of expertise.  The
participants will be able to keep working on them after the tutorials,
with informal help from the experts whenever needed.

Many such worksheets have been crafted at the occasion of previous
such meetings, and this workshop will be the occasion to polish them,
design new ones, and possibly submit some to become
\href{http://software-carpentry.org/lessons/}{Software Carpentry
  lessons}.

Here are some tentative titles:
\begin{itemize}
\item Introduction to Sage
\item Introduction to Mathemagix
\item Introduction to GAP
\item Introduction to Pari/GP
\item Programming in Python
\item Collaborative software development (git, ...)
\item Jupyter notebooks and reproducible research
\end{itemize}

\subsection{Round tables}

We are considering the organization of a couple round tables to run
debates and discussions among many participants on topics such as:

\begin{itemize}
\item Women in ... (tentatively led by Marie-Françoise Roy)
\end{itemize}

\subsection{Collaborative work and coding sprints}

A lot of free time will be left each day, especially at the end of the
week, for collaborative work. To support the self-organization of this
time, there will be regular plenary discussions where participants
will briefly expose the projects they want to work on and call for
collaborators. Typical projects can be:
\begin{itemize}
\item I would like to train myself further on .. by going through
  worksheet ...; who would like to join?
\item I would like to learn more about ...; who else would be
  interested? Could we have a tutorial?
\item For my research, I am writing a program that ...; I would need
  expert help on using ... for this.
\item We want to add feature ... to ...; who wants to join?
\end{itemize}

A list of projects (and related progress report), will be maintained
on the conference web page (typically through a wiki). Participants
will be able to start suggest projects in the months before the
conference, and keep adding more during the conference.

\end{document}
