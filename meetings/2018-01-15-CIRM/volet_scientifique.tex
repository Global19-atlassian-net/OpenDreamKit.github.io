\documentclass[12pt]{amsart}
\usepackage[utf8]{inputenc}
\usepackage{ae,aecompl,aeguill}	% pour utiliser << et >>
\usepackage[english]{babel}
\usepackage[french]{babel}
\usepackage{times}
\usepackage[babel=true,kerning=true]{microtype}
\usepackage[pdfborder={0 0 0}]{hyperref}
\usepackage{xspace}

\usepackage[left=3cm,right=3cm,top=3cm, bottom=3cm]{geometry}

\usepackage[parfill]{parskip}

\makeatletter
\def\subsection{\@startsection{subsection}{2}%
  \z@{.3\linespacing\@plus.5\linespacing}{.1\linespacing}%
  {\normalfont\bfseries}}

\makeatother

\newcommand{\scienceproject}{\mbox{\textsc{SCIEnce}}\xspace}

\newcommand{\OOMMFNB}{OOMMF-NB\xspace}
\newcommand{\VRE}{VRE\xspace}
\newcommand{\VREs}{VRE\xspace}

\newcommand{\software}[1]{\textsc{#1}\xspace}

\newcommand{\GAP}{\software{GAP}}
\newcommand{\HPCGAP}{\software{HPC-GAP}}
\newcommand{\libGAP}{\software{libGAP}}
\newcommand{\Singular}{\software{Singular}}
\newcommand{\Sage}{\software{Sage}}
\newcommand{\SageCombinat}{\software{Sage-Combinat}}
\newcommand{\MuPADCombinat}{\software{MuPAD-Combinat}}
\newcommand{\Sphinx}{\software{Sphinx}}
\newcommand{\SCSCP}{\software{SCSCP}}
\newcommand{\Python}{\software{Python}}
\newcommand{\IPython}{\software{IPython}}
\newcommand{\Jupyter}{\software{Jupyter}}
\newcommand{\Cython}{\software{Cython}}
\newcommand{\Pythran}{\software{Pythran}}
\newcommand{\Numpy}{\software{Numpy}}
\newcommand{\Pari}{\software{PARI}}
\newcommand{\PariGP}{\software{PARI/GP}}
\newcommand{\libpari}{\software{libpari}}
\newcommand{\GP}{\software{GP}}
\newcommand{\GPtoC}{\software{GP2C}}
\newcommand{\Linbox}{\software{LinBox}}
\newcommand{\LMFDB}{\software{LMFDB}}
\newcommand{\OpenEdX}{\software{OpenEdX}}
\newcommand{\Linux}{\software{Linux}}
\newcommand{\LATEX}{\software{\LaTeX}}
\newcommand{\SMC}{\software{SageMathCloud}}
\newcommand{\Simulagora}{\software{Simulagora}}
\newcommand{\KANT}{\software{KANT}}
\newcommand{\Magma}{\software{Magma}}
\newcommand{\Mathematica}{\software{Mathematica}}
\newcommand{\Maple}{\software{Maple}}
\newcommand{\Matlab}{\software{Matlab}}
\newcommand{\MuPAD}{\software{MuPAD}}
\newcommand{\MPIR}{\software{MPIR}}
\newcommand{\Arxiv}{\software{arXiv}}
\newcommand{\Givaro}{\software{Givaro}}
\newcommand{\fflas}{\software{fflas}}
\newcommand{\MathHub}{\software{MathHub}}
\newcommand{\xcas}{\software{Giac/Xcas}}
\newcommand{\Mathemagix}{\software{Mathemagix}}
\newcommand{\FindStat}{\software{FindStat}}
\newcommand{\OSCAR}{\software{OSCAR}}
\newcommand{\Nauty}{\software{Nauty}}
\newcommand{\Bliss}{\software{Bliss}}
\newcommand{\ACE}{\software{ACE}}
\newcommand{\KBMAG}{\software{KBMAG}}

\newcommand\DKS{\ensuremath{\mathcal{DKS}}\xspace}
\newcommand\OEIS{\ensuremath{\mathcal{OEIS}}\xspace}

\newcommand{\ODK}{\href{http://opendreamkit.org}{OpenDreamKit}\xspace}

%%% Local Variables: 
%%% mode: latex
%%% TeX-master: "proposal"
%%% End: 


\begin{document}

\title{Calcul Mathématique Libre 2019:\\Volet scientifique}
\maketitle
\thispagestyle{empty}

\renewcommand{\contentsname}{}
\section*{Contents}
\tableofcontents

\section{Scientific context}

\subsection{The ecosystem of open-source mathematical software}

From their earliest days, computers have been used in pure
mathematics, either to make tables, to prove theorems (famously the
four colour theorem) or, as with the astronomer's telescope, to
explore new theories. Computer-aided experiments, and the use of
databases relying on computer calculations such as the Small Groups
Library in GAP, the Modular Atlas in group and representation theory,
or the $L$-functions and Modular Forms Database (\LMFDB, see later),
are part of the standard toolbox of the pure mathematician, and
certain areas of mathematics completely depend on it. Computers are
also increasingly used to support collaborative work and education.

The last decades witnessed the emergence of a wide ecosystem of
open-source tools to support research in pure mathematics. This ranges
from specialized to general purpose computational tools such as \GAP,
\PariGP, \Linbox, \MPIR, \Sage, or \Singular, to online databases
like the \LMFDB, not to mention online services like Wikipedia,
\Arxiv, or MathOverflow. A great opportunity is the rapid emergence of
key technologies, and in particular the \Jupyter (previously \IPython)
platform for interactive and exploratory computing which targets all
areas of science. This has proven the viability and power of
collaborative open-source development models, by users and for users,
even for delivering general purpose systems targeting a large public
(researchers, teachers, engineers, amateurs, \ldots). A Horizon 2020 funded
 project named OpenDreamKit aims at developing all these opensource
tools so that they can form a flexible Virtual Research Environment toolkit. 
When the OpenDreamKit consortium reaches its goals, it will be a breakthrough
concerning the methods for research and teaching that need mathematics tools.
Since this conference will take place at the end of the four-year long OpenDreamKit
project, the talks to be presented and the tutorials to be organised 
will reflect the quality of research OpenDreamKit will have reached by
then. Furthermore it will be the occasion to present the final flexible
VRE to a large audience.


An exemplary success in the last decade is that of \Sage,
a free general purpose open-source mathematics
software system licensed under the Gnu Public License (GPL) whose mission is to create a
viable free open source alternative to Magma, Maple, Mathematica and
Matlab. It has been developed since 2005 by a growing worldwide community of
about 150 researchers and teachers. It builds on top of many existing
open-source packages, including NumPy, SciPy, matplotlib, Sympy,
Maxima, and the aforementioned ones, all accessible from a
Python-based library containing itself many unique mathematical
features. \Sage can also be used as batch program, or through the
\Jupyter interactive computing interface (command-line or graphical),
or \href{cloud.sagemath.org}{on the cloud}.

Thanks to this, \Sage is regularly used in universities, both for
research and education purposes.  An indicator of its success in the French
system is that \Sage has been since 2014 in the shortlist of official
software for the oral examinations of the Agrégation de Mathématiques
(the nation-wide recruiting competition for high school teachers), and
that since 2015 this list has been containing only open source software.

Following this example, one can only imagine the impact the
OpenDreamKit VRE could have for the academic community in France,
in Europe and beyond.

\subsection{Upcoming challenges}

Some of the upcoming major challenges are:
\begin{itemize}
\item Lower the entry barrier, in particular via \textbf{unified user
    interfaces}, and \textbf{Virtual Research Environments} that
  groups of users can setup to collaborate on data, software,
  computations, or knowledge;
\item Further enable \textbf{computations involving multiple systems},
  as transparently as possible;
\item Keep the development efforts manageable as the size and
  complexity of software systems increase;
\item Train a new generation of users and developers.
\end{itemize}

A key step is to strengthen collaborations \emph{between} the various
communities, in order to:
\begin{itemize}
\item Seek for opportunities for collaboration or outsourcing of
  components to save on development efforts;
\item Share expertise and best practices;
\item Improve cross-systems development workflows.
\end{itemize}

\section{Conference goals and public}

\subsection{Goals: community building and training}

The conference goals can be summarized as follows:

\begin{itemize}
\item Bringing together the community of users and developers of the
  ecosystem of (open source) (pure) mathematics software;
\item Give newcomers as well as specialists an overview of this
  ecosystem: what are the existing software systems, what sorts of
  problems can be solved thanks to them, how they are developed and
  by whom;
\item Train a new generation of users (and the current generations as
  well) on using this ecosystem to solve their own problems;
\item Encourage people to get involved and contribute, especially
  women and junior researchers, in particular by exposing them with role models;
\item Share perspectives and best practices;
\item Build a joint vision and
  seek venues for tighter cooperation.
\end{itemize}

\subsection{Public and prerequisites}

According to the above objectives, the public is meant to be diverse:
\begin{itemize}
\item Newcomers, and especially \textbf{graduate students and young
    researchers}, that need to train themselves with computational
  tools for their teaching or research;
\item More advanced users that want to get an overview on the
  ecosystem or benefit from expert advice;
\item Contributors of the different systems of various levels of
  expertise that want to share their knowledge, learn more, and
  participate to coding sprints, in particular on cross-system
  features.
\end{itemize}

It is to be expected that most participants will have a strong
background in pure mathematics or computer science, although
engineers, teachers, or researchers in neighboring fields (e.g. physics)
having a need for computational mathematics tools will be
welcome.

Generally speaking there will be little prerequisites besides a strong
desire to learn, get one's hands dirty, work in team, and possibly get
involved. It is expected also that the participants will come with
their own laptops. The software will be available either for direct
installation or for remote usage through the web.

\section{More context}

\subsection{The emergence of the OpenDreamKit toolkit}

\href{http://opendreamkit.org}{OpenDreamKit} is a Horizon 2020
European Research Infrastructure project (\#676541) funded under the
H2020-EINFRA-2015-1 call, that is running for four years, starting in
September 2015 and ending August 2019. It provides substantial funding
to the development open source computational mathematics ecosystem,
and in particular popular tools
such as \Linbox, \MPIR, \Sage, \GAP, \Pari/GP, \LMFDB, \Singular,
\MathHub, and the \IPython/\Jupyter interactive computing environment.

As it was previously shortly mentioned, from this ecosystem,
\ODK will deliver a flexible toolkit enabling
research groups to set up Virtual Research Environments, customised to
meet the varied needs of research projects in pure mathematics and
applications, and supporting the full research life-cycle from
exploration, through proof and publication, to archival and sharing of
data and code. This ecosystem is to be widely presented at the
CIRM premises.

The \ODK  consortium consists of core European developers of the
aforementioned systems for pure mathematics, and extending toward the
numerical community, and in particular the \Jupyter  community, to work
together on joint needs.

\textbf{Community building, dissemination and training is at the heart of \ODK}. From the
beginning of the design of the proposal, back in 2014 (during ALEA
2014 at CIRM!), it was planned to fund yearly large meetings to reach
out to the larger community, disseminate the outcome, and train new
users and developers. Besides CIRM in 2019, two other meetings will have had
taken place: one at the ICMS (Edinburgh) in January 2017 and  one in Dagstuhl in 2018.

% Yet some critical long term investments, in particular on the
% technical side, are in order to boost the productivity and lower the
% entry barrier:

\subsection{The \emph{Developers Days} tradition}

The success of any research software or service is strongly linked to
its ability to attract and retain a large number of users. The
different communities (\Sage, \GAP, \Pari, \Singular, \Jupyter, ...)
have each developed sustainable networks. For example, \Sage has
accumulated thousands of users in under 10 years.

This has been achieved thanks to a very strong community building
philosophy, especially through the organization of ``\emph{Developer
  Days}'', thematic workshops aimed at core developers and newcomers,
focused on improving the software.

For example, the first
``\href{https://wiki.sagemath.org/Workshops}{\Sage Days}'' was held in
2006 with 10 participants; to date there has been at least 85 of them,
including about ten in 2016. Those workshops are typically week-long
and cover a wide variety of topics, ranging from focused workshops
gathering a dozen developers for coding sprints on specific
mathematical or technical features to training workshops with more
than 80 people.

The \PariGP community uses a similar format for its
``\href{http://pari.math.u-bordeaux.fr/ateliers.html}{Ateliers
  \PariGP}'', started in 2004, and held yearly since 2012. More
recently, the GAP community has started its series of
``\href{http://gapdays.de/}{GAP Days}'', including joint GAP/\Sage
days.

To cite just a few of the recent or upcoming developers days:

\begin{itemize}
\item \Sage Education Days 6 (June 16-18, 2014, University of
  Washington, Seattle)
\item \Sage Days 82 (January 9-13, 2017, Paris): Women in Sage
\item Atelier PARI/GP 2017 (January 9-13, 2017, Lyon)
\item \Sage/GAP Days 85  (March 04-08, 2017. Cernay, Paris area, France):
  Packaging, portability, documentation tools
\item \Sage Days 79 (November 21-25, 2016, Jerusalem, Israel): Combinatorics
\end{itemize}

All those workshops have in common to be highly hands on, with lots of
room for tutorials and collaborative work, and a strong dynamic
(project sessions, status reports, ...) to get everyone involved.

\subsection{Previous \Sage Days at CIRM}

One particularly successful workshop,
\href{https://www.lirmm.fr/arith/wiki/MathInfo2010/SageDays}{\Sage
  Days 20}, was organized at CIRM in 2010, at the occasion of the
thematic month
\href{https://www.lirmm.fr/arith/wiki/MathInfo2010/}{Math-Info}. This
workshop had a double focus on training and research. It was one of
the largest \Sage Days, with about 80 participants, many of which
became at this occasion regular \Sage users, if not contributors. One
of the many outcome was a strong initial impetus to what was to become
the first book about \Sage: \href{http://sagebook.gforge.inria.fr/}{Calcul Mathématique avec
Sage}, in French and under an open
source license. Also a lot of training material was written during the
workshop and reused extensively in followup workshops; a large chunk
of this material was integrated in the \Sage official documentation in
the form of thematic tutorials.

The commodities at CIRM turned out just right for the purpose, and
since then the proposers have wished to organize another one.

\subsection{Composition of the committees, invited speakers, and gender issues}

All the members of the Organization and Scientific committees have a
strong experience with \Sage Days or similar events. For example,
Nicolas Thiéry attended 20+ \Sage Days, was main organizer or
coorganizer of more than a dozen of them, and invited speaker in many
of the others, including that at CIRM. They are also involved in the
OpenDreamKit project and present a variety of perspectives.

Alas, our field suffers very badly from the lack of gender parity (5
to 10\% of female researchers). Despite our best efforts this is reflected by the
composition of the committees. However the community is struggling
to attract women into the opensource development community. Viviane Pons
organised in January 2017 the first \href{https://wiki.sagemath.org/days82}{Women in Sage}
in the Paris region. The enthusiasm of all participants is a sign showing
that further actions must be taken.

As a result we have focused on showcasing female role models in the keynote
speakers (Marie Françoise Roy and Anne Schilling and we are seeking
for another one), and for maintaining a proper ratio among tutorial
leaders. We will also take measures to attract and promote female
participants, and in particular PhD students and Postdocs.

\section{Program}

The program will consist of keynote talks (one or two per day), many
hands on tutorials, some round tables, and a lot of free time for
collaborative work and coding sprints.

\subsection{Keynote talks}

We are aiming at 5-6 one-hour keynote talks. The goal is to deliver a
variety of perspectives on the ecosystem of open source software for
mathematics. At this point we have five confirmed speakers:

\begin{description}
\item[Marie-Françoise Roy] Historical perspective on contributions of
  researchers and teachers to (open source) mathematical software
\item[Joris VanDerHoeven] Design of general purpose mathematical software: \Mathemagix (and TeXmacs)
\item[Max Horn] Collaborative software development in a large system (\GAP)
\item[Fernando Perez] Scientific and interactive computing, user
  interfaces (e.g. \Jupyter) and community
\item[Anne Schilling] Experimental mathematics
\end{description}

We are aiming at complementing those with speakers presenting the
developers perspective from a large system like GAP and a specialized
library. We have a short list of speakers for those, but decided to
postpone the decision and invitations, in order to keep some
flexibility to accommodate for potential evolutions in the coming
months as we reach toward neighbor communities (e.g. proof systems).

\subsection{Hands-on tutorials}

We will reserve 6-8 slots for hands-on tutorial on the various
software of the ecosystem. Each tutorial will be delivered by experts
(typically one leader and several helpers). After a brief overview,
the participants will be guided through a collection of exercise
worksheets, designed to accommodate various levels of expertise.  The
participants will be able to keep working on them after the tutorials,
with informal help from the experts whenever needed.

Many such worksheets have been crafted over time at the occasion of
previous such meetings (including several of
\href{http://doc.sagemath.org/html/en/thematic_tutorials/}{Sage's
  thematic tutorials}). This workshop will be the occasion to polish them,
design new ones, and possibly submit some to become
\href{http://software-carpentry.org/lessons/}{Software Carpentry lessons}.

Here are some tentative titles:
\begin{itemize}
\item Software installation
\item Jupyter notebooks and reproducible research
\item Introduction to GAP
\item Introduction to Linbox
\item Introduction to Mathemagix
\item Introduction to PARI/GP
\item Introduction to Sage
\item Introduction to Singular
\item Introduction to XCas
\item Introduction to Nemo (OSCAR)
\item Programming in Python
\item Collaborative software development (git, ...)
\end{itemize}

\subsection{Round tables and special event for teaching}

We are considering the organization of a few round tables to run
debates and discussions on topics such as:

\begin{itemize}
\item Women in open source mathematics software (tentatively led by
  Marie-Françoise Roy);
\item Development models, best practices, and funding for open source
  mathematics software development;
\item Software for teaching;
\item ...
\end{itemize}

We will also reach toward local high-school, \textit{classes préparatoires},
and university teachers and, pending enough interest, organize one
half-day event focused on teaching (as in the previous \Sage Days at
CIRM).

\subsection{Collaborative work and coding sprints}

A lot of free time will be reserved each day, especially at the end of
the week, for collaborative work. To support the self-organization of
this time, there will be regular plenary discussions where
participants will briefly expose the projects they want to work on and
call for collaborators. Typical projects will be of the form:
\begin{itemize}
\item I would like to train myself further on XXX by going through
  worksheet YYY; who would like to join?
\item I would like to learn more about XXX; who else would be
  interested? Who could deliver a tutorial?
\item For my research, I am writing a program that XXX; I would need
  expert help on using YYY for this.
\item We want to add feature XXX to YYY; who wants to join?
\end{itemize}

A list of projects, and related progress report, will be maintained on
the conference web page (typically through a wiki). Participants will
be able to start suggesting projects in the months before the conference,
and keep adding more during the conference.

\subsection{Joint dynamics with JNCF 2019 and ALEA 2019}

The communities involved in open source mathematical software are
strong in Europe. There are in particular a good overlaps with the
French Computer Algebra and ALEA communities.  Therefore, we are in
contact with the organizing committee of the ``Journées Nationales du
Calcul Formel 2019'', and we have prepared the program so to attract
researchers from both communities.

That justifies our first choice for the dates: CIRM has here an
opportunity to boost participation to more than one event by planning
them close by, thus lowering opportunity and travel costs, and
strongly increasing international attractiveness.

\end{document}
