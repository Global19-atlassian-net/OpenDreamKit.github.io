\documentclass[12pt]{amsbook}
\usepackage[utf8]{inputenc}
\usepackage{ae,aecompl,aeguill}	% pour utiliser << et >>
\usepackage[francais]{babel}
\frenchbsetup{og=«, fg=»}
\usepackage{times}
\usepackage[babel=true,kerning=true]{microtype}
\usepackage{hyperref}
\usepackage{xspace}

\newcommand{\scienceproject}{\mbox{\textsc{SCIEnce}}\xspace}

\newcommand{\OOMMFNB}{OOMMF-NB\xspace}
\newcommand{\VRE}{VRE\xspace}
\newcommand{\VREs}{VRE\xspace}

\newcommand{\software}[1]{\textsc{#1}\xspace}

\newcommand{\GAP}{\software{GAP}}
\newcommand{\HPCGAP}{\software{HPC-GAP}}
\newcommand{\libGAP}{\software{libGAP}}
\newcommand{\Singular}{\software{Singular}}
\newcommand{\Sage}{\software{Sage}}
\newcommand{\SageCombinat}{\software{Sage-Combinat}}
\newcommand{\MuPADCombinat}{\software{MuPAD-Combinat}}
\newcommand{\Sphinx}{\software{Sphinx}}
\newcommand{\SCSCP}{\software{SCSCP}}
\newcommand{\Python}{\software{Python}}
\newcommand{\IPython}{\software{IPython}}
\newcommand{\Jupyter}{\software{Jupyter}}
\newcommand{\Cython}{\software{Cython}}
\newcommand{\Pythran}{\software{Pythran}}
\newcommand{\Numpy}{\software{Numpy}}
\newcommand{\Pari}{\software{PARI}}
\newcommand{\PariGP}{\software{PARI/GP}}
\newcommand{\libpari}{\software{libpari}}
\newcommand{\GP}{\software{GP}}
\newcommand{\GPtoC}{\software{GP2C}}
\newcommand{\Linbox}{\software{LinBox}}
\newcommand{\LMFDB}{\software{LMFDB}}
\newcommand{\OpenEdX}{\software{OpenEdX}}
\newcommand{\Linux}{\software{Linux}}
\newcommand{\LATEX}{\software{\LaTeX}}
\newcommand{\SMC}{\software{SageMathCloud}}
\newcommand{\Simulagora}{\software{Simulagora}}
\newcommand{\KANT}{\software{KANT}}
\newcommand{\Magma}{\software{Magma}}
\newcommand{\Mathematica}{\software{Mathematica}}
\newcommand{\Maple}{\software{Maple}}
\newcommand{\Matlab}{\software{Matlab}}
\newcommand{\MuPAD}{\software{MuPAD}}
\newcommand{\MPIR}{\software{MPIR}}
\newcommand{\Arxiv}{\software{arXiv}}
\newcommand{\Givaro}{\software{Givaro}}
\newcommand{\fflas}{\software{fflas}}
\newcommand{\MathHub}{\software{MathHub}}
\newcommand{\xcas}{\software{Giac/Xcas}}
\newcommand{\Mathemagix}{\software{Mathemagix}}
\newcommand{\FindStat}{\software{FindStat}}
\newcommand{\OSCAR}{\software{OSCAR}}
\newcommand{\Nauty}{\software{Nauty}}
\newcommand{\Bliss}{\software{Bliss}}
\newcommand{\ACE}{\software{ACE}}
\newcommand{\KBMAG}{\software{KBMAG}}

\newcommand\DKS{\ensuremath{\mathcal{DKS}}\xspace}
\newcommand\OEIS{\ensuremath{\mathcal{OEIS}}\xspace}

\newcommand{\ODK}{\href{http://opendreamkit.org}{OpenDreamKit}\xspace}

%%% Local Variables: 
%%% mode: latex
%%% TeX-master: "proposal"
%%% End: 


\begin{document}

\title{Calcul Mathématique Libre, CIRM 2018: Abstract}

The last decades have witnessed the emergence of a full ecosystem of
open source software for (pure) mathematics, developed by overlapping
international communities of researchers, teachers, engineers and
amateurs. This ranges from specialized libraries (e.g. \MPIR, \Linbox)
to thematic systems (e.g. \GAP, \Pari, \Singular, xcas) to general
purpose systems (e.g. Mathemagix, \Sage), via online databases
(e.g. the OEIS, \MathHub, or the LMFDB). This is part of the greater
movement for Open and Reproducible Science, and is supported by the
advancement of cross-discipline tools like the interactive computing
environment \Jupyter.


The main goal of this conference, cofunded by the H2020 European
E-Infrastructure project \href{opendreamkit.org}{OpenDreamKit}, is
community building and training:
\begin{itemize}
\item Bringing together the various communities of users and
  developers of the ecosystem of (open source) (pure) mathematics
  software.
\item Give newcomers as well as experts an overview of this ecosystem:
  what the existing software systems are, what they can compute or solve, how
  they are developed and by whom, what the success stories and
  difficulties are.
\item Train newcomers as well as experts on using this ecosystem to
  solve their own problems.
\item Share perspectives and best practices, build a joint vision, and
  seek venues for tighter cooperation.
\item Encourage participants to get involved,especially young and
  women, in particular by showcasing role models.
\end{itemize}


Following a long trend of highly productive workshops within the
various communities (e.g. the Sage Days series), this conference will
consist of:
\begin{itemize}
\item 5-6 keynote talks delivering a variety of perspectives on the
  ecosystem.
\item Hands-on tutorials run by experts of the various systems.
\item Plenty of free time for interactions and collaborative work,
  self-organized through regular project sessions and progress
  reports.
\end{itemize}

\end{document}

