\documentclass[12pt]{amsbook}
\usepackage[utf8]{inputenc}
\usepackage{ae,aecompl,aeguill}	% pour utiliser << et >>
\usepackage[francais]{babel}
\frenchbsetup{og=«, fg=»}
\usepackage{times}
\usepackage[babel=true,kerning=true]{microtype}
\usepackage{hyperref}
\usepackage{xspace}

\newcommand{\scienceproject}{\mbox{\textsc{SCIEnce}}\xspace}

\newcommand{\OOMMFNB}{OOMMF-NB\xspace}
\newcommand{\VRE}{VRE\xspace}
\newcommand{\VREs}{VRE\xspace}

\newcommand{\software}[1]{\textsc{#1}\xspace}

\newcommand{\GAP}{\software{GAP}}
\newcommand{\HPCGAP}{\software{HPC-GAP}}
\newcommand{\libGAP}{\software{libGAP}}
\newcommand{\Singular}{\software{Singular}}
\newcommand{\Sage}{\software{Sage}}
\newcommand{\SageCombinat}{\software{Sage-Combinat}}
\newcommand{\MuPADCombinat}{\software{MuPAD-Combinat}}
\newcommand{\Sphinx}{\software{Sphinx}}
\newcommand{\SCSCP}{\software{SCSCP}}
\newcommand{\Python}{\software{Python}}
\newcommand{\IPython}{\software{IPython}}
\newcommand{\Jupyter}{\software{Jupyter}}
\newcommand{\Cython}{\software{Cython}}
\newcommand{\Pythran}{\software{Pythran}}
\newcommand{\Numpy}{\software{Numpy}}
\newcommand{\Pari}{\software{PARI}}
\newcommand{\PariGP}{\software{PARI/GP}}
\newcommand{\libpari}{\software{libpari}}
\newcommand{\GP}{\software{GP}}
\newcommand{\GPtoC}{\software{GP2C}}
\newcommand{\Linbox}{\software{LinBox}}
\newcommand{\LMFDB}{\software{LMFDB}}
\newcommand{\OpenEdX}{\software{OpenEdX}}
\newcommand{\Linux}{\software{Linux}}
\newcommand{\LATEX}{\software{\LaTeX}}
\newcommand{\SMC}{\software{SageMathCloud}}
\newcommand{\Simulagora}{\software{Simulagora}}
\newcommand{\KANT}{\software{KANT}}
\newcommand{\Magma}{\software{Magma}}
\newcommand{\Mathematica}{\software{Mathematica}}
\newcommand{\Maple}{\software{Maple}}
\newcommand{\Matlab}{\software{Matlab}}
\newcommand{\MuPAD}{\software{MuPAD}}
\newcommand{\MPIR}{\software{MPIR}}
\newcommand{\Arxiv}{\software{arXiv}}
\newcommand{\Givaro}{\software{Givaro}}
\newcommand{\fflas}{\software{fflas}}
\newcommand{\MathHub}{\software{MathHub}}
\newcommand{\xcas}{\software{Giac/Xcas}}
\newcommand{\Mathemagix}{\software{Mathemagix}}
\newcommand{\FindStat}{\software{FindStat}}
\newcommand{\OSCAR}{\software{OSCAR}}
\newcommand{\Nauty}{\software{Nauty}}
\newcommand{\Bliss}{\software{Bliss}}
\newcommand{\ACE}{\software{ACE}}
\newcommand{\KBMAG}{\software{KBMAG}}

\newcommand\DKS{\ensuremath{\mathcal{DKS}}\xspace}
\newcommand\OEIS{\ensuremath{\mathcal{OEIS}}\xspace}

\newcommand{\ODK}{\href{http://opendreamkit.org}{OpenDreamKit}\xspace}

%%% Local Variables: 
%%% mode: latex
%%% TeX-master: "proposal"
%%% End: 


\begin{document}

\title{Calcul Mathématique Libre, CIRM 2018: Résumé}
\maketitle
Les dernières décennies ont vu émerger un écosystème complet de logiciels libres pour les mathématiques (fondamentales), développé par une  superposition de communeautés internationales de chercheurs, d'enseignants, d'ingénieurs et d'amateurs. Cela va de bibliothèques spécialisées (e.g. \MPIR, \Linbox), à des systèmes thématiques (e.g. \GAP, \Pari, \Singular, xcas) jusqu'à des systèmes  modulaires d'utilisation (ex: Mathemagix, \Sage), en pasant par des bases de données en ligne (ex: OEIS, \MathHub, ou LMFDB). Cet écosystème fait partie d'une plus grande tendance vers une science ouverte et reproductible,  et il est soutenu par l'avancement d'outils pluridisciplinaires tel que l'environnement d'informatique intéractif \Jupyter.


Cette conférence, cofinancée par le projet H2020 European Infrastructure  \href{opendreamkit.org}{OpenDreamKit}, a pour objectif principal la formation et le renforcement de la communeauté:

\begin{itemize}
\item Rassembler les diverses communeautés d'utilisateurs et de développeurs de l'écosystème de logiciels (libres) en mathématiques (fondamentales).
\item Donner aux nouveaux arrivants ainsi qu'aux experts une vue d'ensemble de cet écosystème: les logiciels existants, ce qu'ils peuvent calculer ou résoudre, comment ils sont développés et par qui, quelles sont les \textit{success stories} et les difficultés.
  \item Former aussi bien les nouveaux arrivants que les experts à l'usage de cet écosystème pour résoudre leurs propres problèmes.
\item Partager les points de vue et les bonnes pratiques, construire une vision commune et rechercher les moyens pour une coopération plus étroite.
\item Encourager les participants à s'impliquer, spécifiquement les jeunes et les femmes, en particulier grâce à la présentation de personnes modèles.
\end{itemize}


Suivant une tendance de longue date d'ateliers extrêmement productifs au sein des diverses communeautés (ex: les Sage Days), cette conférence sera organisée comme suit:
\begin{itemize}
\item 5 à 6 conférenciers exposeront leur point de vue sur l'écosystème (un par jour)
\item Sessions pratiques dirigées par les experst sur les différents systèmes
\item Beaucoup de temps libre propice aux intéractions et au travail collaboratif, qui s'organiseront d'elles-mêmes à travers des rapports de progression et sessions de projet réguliers.
\end{itemize}

Partout en Europe et particulièrement en France, les communeautés impliquées sont dynamiques. Nous nous coordonnons avec les Journées Nationales du Calcul Formel 2018 dans l'espoir d'organiser notre conférence et leur événement l'un après l'autre afin d'encourager les participants à assister aux deux, et ainsi de renforcer notre attractivité pour les personnes éloignées.

\end{document}

