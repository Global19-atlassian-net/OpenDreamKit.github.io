\documentclass{paper}

\usepackage[utf8]{inputenc}
\usepackage[T1]{fontenc}
\usepackage[french]{babel}
\usepackage{charter}
\usepackage[pdfborder={0 0 0}]{hyperref}

\title{List of confirmed keynote speakers}
\subtitle{Calcul Mathématique Libre 2018}
\date{}

\newcommand{\orateur}[3]{%
  \section*{%
    #1 {\small(#2)}\nopagebreak\\
    #3}
}

\begin{document}

\maketitle
\thispagestyle{empty}

We list here the main invited speakers that have confirmed their
intention to participate and deliver a one hour keynote talk.

% Des exposés de recherche seront prévus en fonction des sujets
% d’actualité dans deux ans.  Ces exposés seront donnés en priorité par
% des étudiants en thèse ou des post-doctorants, dont il est difficile
% de lister les noms à ce jour.

% \emph{Tous les conférenciers cités dans ce document ont confirmé leur participation aux journées.}

\orateur
{Marie-Françoise Roy}
{Emeritus professor, University of Rennes 1}
{A historical perspective on contributions of researchers and teachers
  to (open source) mathematical software}

Marie-Françoise Roy is a French mathematician noted for her work in
real algebraic geometry. She advised 23 students and (co)authored
several books including ``Algorithms in real algebraic geometry'' and
several dozens of scientific papers. Marie-Françoise Roy is highly
involved in the animation of the mathematics community, in particular
in the promotion of Women and African mathematicians. Among other
things, she was president of the Société Mathématique de France from
2004 to 2007 and is currently president of the Committee for Women in
Mathematics of the International Mathematics Union.

\medskip

Page personnelle : \url{https://perso.univ-rennes1.fr/marie-francoise.roy/}


\end{document}
