\documentclass{beamer}

\usetheme[faculty=we,language=english,framenumber,totalframenumber]{UniversiteitGent}

\title{Ghent University in OpenDreamKit}
\author{Jeroen Demeyer}

\begin{document}

\begin{frame}
  \titlepage
\end{frame}

\begin{frame}
\frametitle{UGent added to OpenDreamKit}

\begin{itemize}
\item Jeroen Demeyer applied for a ``ing\'enieur de recherche'' position
at Paris-Sud to work for OpenDreamKit.
\item Jeroen was recruted 1 March 2016,
but there were a lot of administrative difficulties because he did not want to move to France.
\item To solve this, it was decided to add UGent as partner of OpenDreamKit,
starting at 1 July 2016 (hopefully).
\item UGent has a single participant, Jeroen Demeyer,
who will work $30.5\,\text{PM}$ at UGent for OpenDreamKit
(after having worked $2\,\text{PM}$ at Paris-Sud).
\end{itemize}
\end{frame}

\begin{frame}
\frametitle{Who is Jeroen Demeyer?}

\begin{itemize}
\item SageMath developer since July 2010 (Sage Days 23 in Leiden).
Main contributions:
\begin{itemize}
\item Build system.
\item PARI interface.
\item Coercion model, core arithmetic and comparison infrastructure.
\item Lines of code written for Sage according to GitHub: $-125\,115$.
\end{itemize}
\item SageMath release manager from January 2011 (Sage~4.6.1) to December 2013 (Sage~5.13).
\item Main developer of cysignals:
a Cython package for interrupt, signal, error and memory handling.
\item Main developer of pari\_jupyter:
a kernel for the Jupyter Notebook running PARI/GP.
\end{itemize}
\end{frame}

\begin{frame}
\frametitle{D\,3.2: Understand and document SageMathCloud backend code}

Some initial attempts to run SageMathCloud on a personal laptop.

\begin{itemize}
\item There is documentation from upstream on how to do this.

\item Managed to run the SMC server, but starting projects did not work.

\item Various issues reported upstream, no response yet.
\end{itemize}
\end{frame}

\begin{frame}
\frametitle{D\,4.1: Python/Cython bindings for PARI and its integration in Sage}

This deliverable is about splitting off the PARI interface of Sage
to a separate package \texttt{cypari}.

\begin{itemize}
\item This turned out to be much harder than originally anticipated:
the PARI bindings are quite closely tied to Sage.

\item Two most important problems:
\begin{itemize}
\item Interrupt (CTRL-C) and error handling using the \texttt{sig\_on()} mechanism.

\item The coercion model which is needed to do arithmetic between PARI elements
and other Sage elements.
\end{itemize}
\end{itemize}
\end{frame}

\begin{frame}
\frametitle{D\,4.1: Python/Cython bindings for PARI and its integration in Sage}
This deliverable has been split in 4 sub-tasks:
\begin{enumerate}
\item Split off interrupt/signal/error handling to a separate package \texttt{cysignals}:
DONE.

\item Refactor the Sage coercion model. Add better support for non-Sage types,
such that the PARI bindings no longer need the coercion model.
In progress.

\item Split off the remaining Sage-specific parts of the PARI interface
(for example, conversion from/to Sage types).
In progress.

\item Finally, actually split off the PARI bindings from Sage as a new package.
Should be easy once the rest has been done.
\end{enumerate}
\end{frame}

\begin{frame}
\frametitle{D\,4.4: Basic Jupyter interface for GAP, PARI/GP, SageMath, Singular}

First version of Jupyter kernel for PARI/GP is done: Python package \texttt{pari\_jupyter} on PyPI.

\begin{itemize}
\item Supports all GP functions except plotting.

\item Supports history and TAB-completion as in GP.

\item TODO for second version (D\,4.7): syntax highlighting, plotting, break loop.

\item TODO for Jupyter upstream: decide on the proper way to install kernel specs!
\end{itemize}
\end{frame}

\begin{frame}
\frametitle{D\,4.13: Refactorisation of SageMath's Sphinx documentation system}

Big task, but many things have been done:

\begin{itemize}
\item Sphinx in Sage has been upgraded to version 1.4.1 (latest upstream is version 1.4.4).

\item A lot of cleaning up and minor refactoring has been done.

\item Reduce the amount of Sage-specific stuff in docbuilder: a lot of progress
but a lot remains to do.
This is mostly about Cython, so coordination with Python/Cython upstream might be needed.

\item Some work has been done with upstream to reduce the memory footprint of Sphinx.
\end{itemize}
\end{frame}

\end{document}
