\documentclass[12pt]{paper}

\usepackage[utf8]{inputenc}
\usepackage[T1]{fontenc}
\usepackage[french]{babel}
\usepackage[left=3cm,right=3cm,top=1.5cm, bottom=2cm]{geometry}
\usepackage[pdfborder={0 0 0}]{hyperref}

\title{Calcul Mathématique Libre 2019:\\ List of confirmed keynote speakers}
\date{}

\newcommand{\orateur}[3]{%
  \section*{%
    #1 {\small(#2)}\nopagebreak\\
    #3}
}


\begin{document}

\maketitle
\thispagestyle{empty}

We list here the main invited speakers that have confirmed their
intention to participate and deliver a one hour keynote talk.

% Des exposés de recherche seront prévus en fonction des sujets
% d’actualité dans deux ans.  Ces exposés seront donnés en priorité par
% des étudiants en thèse ou des post-doctorants, dont il est difficile
% de lister les noms à ce jour.

% \emph{Tous les conférenciers cités dans ce document ont confirmé leur participation aux journées.}

\orateur
{Marie-Françoise Roy}
{Emeritus professor, University of Rennes 1}
{A historical perspective on contributions of researchers and teachers
  to (open source) mathematical software}

Marie-Françoise Roy is a French mathematician noted for her work in
real algebraic geometry. She advised 23 students and (co)authored
several books including \textit{Algorithms in real algebraic geometry} and
several dozens of scientific papers. Marie-Françoise Roy is highly
involved in the animation of the mathematics community, in particular
in the promotion of female and African mathematicians. Among other
things, she was president of the Société Mathématique de France from
2004 to 2007 and is currently president of the Committee for Women in
Mathematics of the International Mathematics Union.

Personal webpage : \url{https://perso.univ-rennes1.fr/marie-francoise.roy/}

\orateur
{Max Horn}
{Juniorprofessor for algebra and geometry, JLU Gießen}
{Collaborative software development in a large system (GAP)}
{
  Max Horn is a German mathematician originally working in the field of
  algebraic Lie theory, in particular Kac-Moody theory and buildings. After his
  PhD on \textit{Involutions of Kac-Moody group}, he was PostDoc at the TU
  Braunschweig with Bettina Eick, where he worked on various computational
  problems (non-commutative Gröbner bases in group rings, classification of
  small solvable groups, symbolic computations in infinite families) of Lie
  rings and p-group.
  He has become a major developer of the GAP system, and is at the
  forefront of the renewal of its development model, including the
  coorganization of the four first GAP Days.
  In addition, he has been active contributor to a multitude of open source
  projects, and was project leader for two major non-mathematical open source
  projects (Fink and ScummVM) during that time.

Personal webpage : \url{http://www.quendi.de/en/}
}

\orateur
{Fernando Perez}
{Assistant Professor, Department of Statistics, UC Berkeley;
Faculty Scientist at the Data Science and Technology Division of
Lawrence Berkeley National Laboratory}
{The Jupyter project: scientific computing, user interfaces, and community building}

Fernando Perez is a researcher originally working on applied problems
in physics, mathematics and neuroscience and now focusing on the
question of building tools for computational research across all
scientific domains. He is mostly known in the scientific community for
being the creator of the IPython project which had a vast impact for
development in multiple scientific domains: from astronomy to pure
mathematics. More recently, IPython gave birth to the Jupyter project:
an open-source cross-language user interface for data science and
scientific computing. Fernando Perez was awarded --as part of the
Jupyter team-- the 2017 \textit{ACM Software System Award}, and the
2012 \textit{Award for the Advancement of Free Software} for his
contributions to the open source scientific python ecosystem. He is a
founder and a board member of the NumFOCUS foundation as a founding
co-investigator of the Berkeley Institute for Data Science.

Personal webpage : \url{[3]http://fperez.org/}



\orateur
{Anne Schilling}
{Professor, UC Davis California}
{Impact of computer assisted experimentation in combinatorics}

Anne Schilling is a mathematician known for her work in algebraic combinatorics, 
representation theory and mathematical physics. She advised six students and published
dozens of papers in international journals. She widely promotes experimentation 
in mathematics through her own work and also by her contributions to
the software SageMath. With 56 co-authored tickets and more than 100 reviews, she
is an active contributor and valued member of the community. She developed 
the mathematical package on crystal computation allowing connections to number
theory. She was an organizer of the thematic semester \textit{Automorphic Forms, 
Combinatorial Representation Theory and Multiple Dirichlet Series} at ICERM
in 2013 and has also organized many Sage Days conferences often linked to 
combinatorics development.

Personal webpage : \url{https://www.math.ucdavis.edu/~anne/}

\orateur
{Joris Van Der Hoeven}
{Directeur de recherche, CRNS, école Polytechnique}
{General purpose mathematical software design}

Joris Van der Hoeven is a mathematician whose research interests are
asymptotic calculus, transseries and effective complex analysis. 
He is the author of \textit{Transseries and Real Differential Algebra}
published in 2006 by Springer-Verlag and a co-author of \textit{Asymptotic
 Differential Algebra and Model Theory of Transseries} shared on arXiv in 2015.
He is the creator of the software GNU TeXmacs and lead developer
of the software Mathemagix: an open-source computer algebra and analysis system. 
As such he developed and implemented very efficient algorithms for numerically stable 
multi-precision computations.

Personal webpage : \url{http://www.texmacs.org/joris/main/joris.html}

\end{document}
